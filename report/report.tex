\documentclass[polish,a4paper,11pt]{mwart}

\usepackage[polish, english]{babel}
\usepackage[utf8]{inputenc}
\usepackage{polski}
\usepackage[T1]{fontenc}
\usepackage{lmodern}  % zestaw fontów
\usepackage{indentfirst}
\frenchspacing

\usepackage{enumerate}
\usepackage{graphicx}
\usepackage{float}
\usepackage{makecell}
\usepackage{siunitx}
\sisetup{output-decimal-marker = {,}}
\usepackage{icomma}
\let\lll\undefined
\usepackage{amsmath, amssymb, amsfonts}
\usepackage{mathtools}
\usepackage{import}		% wklejanie pdf_tex
\usepackage{xcolor}		% kolory
\usepackage{microtype}

\usepackage{csquotes}
\DeclareQuoteAlias{croatian}{polish}

\usepackage{placeins}	% poprawia float

\let\Oldsection\section
\renewcommand{\section}{\FloatBarrier\Oldsection}

\let\Oldsubsection\subsection
\renewcommand{\subsection}{\FloatBarrier\Oldsubsection}

\let\Oldsubsubsection\subsubsection
\renewcommand{\subsubsection}{\FloatBarrier\Oldsubsubsection}

\AtBeginDocument{
  \renewcommand{\tablename}{Tab.}
  \renewcommand{\figurename}{Rys.}
}

\begin{document}

	\begin{table}[h] %Tabelka
	\centering
		\begin{tabular}{ | c |  >{\centering\arraybackslash}m{5.5cm} | c | }
			\hline
			\makecell{ \textbf{Wydział:} \\ IMiR \\ \textbf{Rok:}~5 \\ Semestr: 2 } &
			\textbf{\large{Niepewność i Monitoring w Wibroakustyce}} &
			\makecell{Data \\ wykonania \\ ćwiczenia: \\ 9.11.2018 r} \\ \hline
			\makecell{\emph{Wykonujący ćw.:} \\ osoba } &
			\large{Nieklasyczne metody statystyczne} &
			\makecell{Nr ćwiczenia: \\ 5} \\ \hline
		\end{tabular}
	\end{table}

  \section{Cel ćwiczenia}
    Celem ćwiczenia było wyznaczenie niepewności standardowej metodą A
    długookresowych wskaźników hałasu dla losowej próby 10-elementowej z 2009
    roku, wykorzystując wnioskowanie bayesowskie.

  \section{Przebieg ćwiczenia}
    Do rozwiązania problemu użyto języka programowania Python. Napisano
    program, który losował 10-elementową próbę z danych pomiarowych ($L_{DWN}$
    oraz $L_N$ ze stacji monitoringu z roku 2009), a następnie na jej podstawie
    wyznaczał rozkład próbkowy wykorzystując jądrowy estymator funkcji gęstości
    prawdopodobieństwa. 


  \section{Wyniki}

  \section{Wnioski}


\end{document}
